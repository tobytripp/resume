\documentclass[11pt,letterpaper]{moderncv}
\moderncvstyle{casual} % casual banking
\moderncvcolor{green}
\usepackage[utf8]{inputenc}

\usepackage[scale=0.75]{geometry}

% Hyperlinks
\usepackage{hyperref}                    % to use hyperlinks
\definecolor{linkcolour}{rgb}{0,0.2,0.6} % hyperlinks setup
\hypersetup{colorlinks,breaklinks,urlcolor=linkcolour, linkcolor=linkcolour}

\firstname {Tobias}
\familyname {Tripp}

\address{3849 N. Damen \#2}{Chicago Illinois}
\mobile {(312) 213-5476}
\email {toby.tripp@gmail.com}

\begin{document}
\maketitle

\section{Profile}
%\cventry{year--year}{Degree}{Institution}{City}{\textit{Grade}}{Description}
\subsection{I Value These Things:}
\cvline{}{
\begin{itemize}%
\item \emph{Fast, dependable, automated tests} that quickly direct the developer to the source of any problem as well as document the system.
\item \emph{Automated build and deployment} for rapid cycles between feature conception and feature delivery.
\item \emph{Iterative design} that grows solutions to complex problems out of simple, verifiable components.
\item \emph{Pair-Programming} to maximize the benefits of peer review and knowledge sharing.
\item \emph{Test-Driven Development} for solid design, complete tests,
  and confident development.
\end{itemize}}

\section{Skills}
\cvitemwithcomment{Proficient}{\small Ruby, Javascript, Java, C/C++}{}
\cvitemwithcomment{Professional Use}{\small Ada, Perl, Tcl/Tk, Php, Soar}
{also Awk and Ksh, but that's a story\ldots}
\cvitemwithcomment{Hobby Use}{\small Prolog, Haskell, Clojure}{}

\section{Experience}


\cventry {December 2010-Present} {Senior Developer}
         {Backstop Solutions Group} {Chicago, Illinois} {} {
Backstop is an award-winning software provider that creates solutions
for a wide variety of fund managers.  I work on the InvestorBridge
platform: a Rails application that integrates with Backstop's Java CRM
platform to provide an information portal for fund managers to
more easily release information to their investors.\newline
While here I've:
\begin{itemize}
  \item shared expertise company-wide in such areas as Agile planning,
    developer testing, Object-Oriented design principles, Javascript
    and Ruby fundamentals, performance analysis and optimization,
    debugging, configuration management, and general software design.
  \item coached developers through the practice of pair-programming.
  \item given a number of classes in these areas.
\end{itemize}
}

\cventry {April 2007-December 2010} {Senior Consultant} {ThoughtWorks} {Chicago, Illinois} {} {
ThoughtWorks is a global software consultancy with a focus on Agile software delivery and client enablement. While at ThoughtWorks, I was involved with the following clients:}
\cvitem {}{
\begin{itemize}
\item \emph{Major Political Party}: Created a high-performance national
  voter lookup service to aid their collaborators in sharing voter
  information and with locating and de-duplicating their large
  database. The service is imple- mented as a Ruby Rack endpoint in
  front of a MongoDB database. Also created a similar service in
  Sinatra for reserving voters with an iPhone canvassing application.
\item \emph{Major Labor Union}: As part of a new program to organize
  and train 10,000 of their members, led a team to create a
  social-networking platform to support collaboration and involvement.
  The site is implemented in Ruby on Rails and tested with RSpec and
  Cucumber. It is deployed on Passenger.
\item \emph{Oracle Mix}: designed, developed, and deployed
  mix.oracle.com in a compressed engagement of 5 weeks and two
  developers, working closely with Oracle Labs to iteratively deliver
  a new customer engagement site for Oracle customers. The site claims
  to be the first enterprise deployment of JRuby on Rails for customer
  consumption.
\item \emph{Sleepy Giant}: during a brief engagement near the
  beginning of Sleepy Giant's business, I helped coach the company
  through requirements gathering and categorization as well as
  coaching the developer through his early practice of Test-Driven
  Development.
\item \emph{Leading VOIP Provider}: worked as a developer on the team
  that rewrote the company's legacy J2EE front- end customer
  subscription system in Ruby on Rails. Following an initial
  production release, the team released to production every eight
  weeks with new features the client had been unable to add to the
  legacy system without introducing intractable regression issues. At
  last count, the site handled more than 200,000 hits a day and
  integrated with 2 legacy databases and more than 20 web services.
\item \emph{A Leading Wholesale Auto Auction}: This was the largest
  Ruby on Rails project in terms of size and people to be undertaken
  in anywhere in the world. The domain involved working on an auction
  site for trading automo biles which included many custom features
  for sellers. The task was to deliver this application as soon as
  possible to help the customer get an edge in the market. Helped on
  this team as a Ruby on Rails developer to build stories and get this
  project finished ahead of time. Involved in TDD with RoR,
  integrating RoR applications with web-services (written in Java), and
  functional integration testing with Selenium and Ruby.
\end{itemize}
}

\cventry {November 2006-January 2007}  {Software Development Consultant}
         {Menlo Innovations} {Ann Arbor, Michigan} {}
{
  Menlo Innovations builds software for a very diverse client set
  using High Tech Anthropology\textregistered \ and Extreme
  Programming.  Projects included a web based system to ease the
  viewing of compound testing results for a large pharmaceutical
  company and software to power a flow-cytometer for a small Ann Arbor
  firm.\newline
  During this brief but valuable engagement, I was able to:
  \begin{itemize}
    \item mentor junior developers in Object Oriented Design and in
      automated testing.
    \item contribute to architecture discussions and decisions.
  \end{itemize}
}

\cventry {July 2005-November 2006} {Software Engineer} {Soar Technology}
         {Ann Arbor, Michigan} {}
{
  Soar Technology develops and researches cognitive software for
  application in training, modeling \& simulation, intelligence
  analysis, command and control, robotics, and medical
  informatics.\newline
  While at Soar, I:
  \begin{itemize}
  \item Developed tools and systems to apply artificial intelligence
    technology to simulation, modeling, and user interface design
    problems.
   \item Created custom IDE applications on top of the Eclipse platform.
   \item Developed and maintained visualization tools, using Java
     Swing and SWT, for visualizing the state and actions of
     artificially intelligent agents. This included mapping, event
     table, and agent interaction visualizations.
   \item Instigated the deployment and use of multi-platform
     continuous integration practices as part of the software
     development process.
   \item Integrated several diverse systems, including an agent
     component in Soar, to model complex social networks.
   \item Advocated for, and helped implement, an open team room to
     facilitate communication and collaboration.
  \end{itemize}
}

\cventry {May 2002-July 2005} {Research Engineer} {Oasis Advanced Engineering}
         {Auburn Hills, Michigan} {}
{
  Oasis develops hardware and software for advanced military training
  systems, including technology that embeds training devices into
  fielded vehicles. These devices allow crews to train while deployed,
  where normally their skills decline.\newline
  While at Oasis, I:
  \begin{itemize}
  \item Participated in the development of a safety-critical base
    operating system. The system was targeted for the Visual Display
    Unit used to monitor a nuclear power plant. The code followed the
    strict guidelines of the Nuclear Regulatory Commission for
    software safety.
  \item Participated in the implementation of several versions of the
    Abrams Common Software Library (ACSL) and Wolverine Common
    Software Library (WCSL). These projects entail the conversion of
    vehicle software for use in all M1A2 Main Battle Tank, M1A2 SEP
    Main Battle Tank, and Wolverine Mobile Bridge Deployment System
    training devices. The functionality of the vehicles' embedded
    operating systems, graphics systems, and communication bus are
    preserved in a UNIX operating system environment.
  \end{itemize}
}

%% \cventry {Summer 2001} {Associate Research Engineer}
%%          {Oasis Advanced Engineering} {Auburn Hills, Michigan} {}
%% {
%%   During this educational summer internship, I:
%%   \begin{itemize}
%%   \item Participated in the initial development of the Wolverine
%%     Common Software Library (WCSL).
%%   \item Developed, as part of a team, the Digital Display Tabletop
%%     Trainer Wolverine variant (D2T2-W).  This version of the
%%     D2T2 system utilizes the Wolverine Common Software Library
%%     to provide a vehicle display trainer on a desktop PC.
%%   \end{itemize}
%% }

\section{Education}
\cventry {2002} {Bachelor of Science, Computer Science with Honors}
         {Michigan State University} {East Lansing, Michigan} {} {}
\cventry {1993} {63H10 Tracked Vehicle Maintenance Training}
         {U.S. Army Ordnance Training Center} {Aberdeen, Maryland} {}
         {}

\section{Military Service}
\cventry {1995-2000} {Michigan Army National Guard}
         {1070th Maintenance Company} {Lansing, Michigan} {} {}
\cventry {1992-1995} {United States Army}
         {1st Infantry Division} {Fort Riley, Kansas} {} {}

\end{document}
